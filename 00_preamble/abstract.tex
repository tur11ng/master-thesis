\chapter*{Περίληψη}
\addcontentsline{toc}{chapter}{Περίληψη}

\pagestyle{plain}

Στην παρούσα διπλωματική εργασία γίνεται εισαγωγή και μελέτη σε διαφορετικά πρωτόκολλα Υπολογισμού Πολλών Μερών (Secure Multi Party Computation ή SMPC), μερικά από τα οποία στη συνέχεια εφαρμόζονται για τη δημιουργία, της πρώτης στη βιβλιογραφία, BLAS βιβλιοθήκης, για Ασφαλή Υπολογισμό Δύο Μερών, BLAS συναρτήσεων Επιπέδου-1 για πραγματικούς αριθμούς κινητής υποδιαστολής μονής ακρίβειας. Πριν ξεκινήσουμε τη μελέτη αυτή, γίνεται μια εισαγωγή στην κρυπτογραφία και στην αποδείξιμη θεωρητική ασφάλεια παρουσιάζοντας όλο το απαραίτητο θεωρητικό υπόβαθρο, μαθηματικό και αλγοριθμικό, που απαιτείται για τη βαθύτερη κατανόηση του αντικειμένου. Στη μελέτη αυτή εστιάζουμε σε σύγχρονα κρυπτογραφικά εργαλεία που χρησιμοποιούνται σήμερα ως συστατικά στοιχεία για την κατασκευή σύνθετων κρυπτογραφικών σχημάτων, δίνοντας περισσότερη έμφαση σε εργαλεία που χρησιμοποιούνται για την κατασκευή σχημάτων Ασφαλούς Υπολογισμού Πολλών Μερών.

\vspace{0.3cm}

\begin{flushleft}
\huge\textbf{Abstract}
\end{flushleft}

On the current diploma thesis we present an introduction and a study on various Secure Multi Party Computation Protocols (SMPC) where some of them are used next to implement the first SMPC BLAS Level-1 library for single precision real floating point numbers. Before this begin this study, we present an introduction to cryptography and provable theoretical security including all the required mathematical and algorithmic background needed for the deeper understand of the subject. On this study we focus on modern cryptographic tools that are used today as a building part to construct complex cryptographic schemes while giving more focus on the set of these tools that are used mostly in the construction of SMPC schemes.