\section{Κρυπτογραφικό Υπόβαθρο}

\subsection{Βασικά κρυπτογραφικά μοντέλα}

\begin{definition}
Generic Group Model
\end{definition}

\begin{definition}
Common Reference String Model
\end{definition}

\begin{definition}
Standard Model
\end{definition}

\begin{definition}
Random Oracle
\end{definition}

\subsection{Βασικοί Ορισμοί}

Οι παράμετροι ασφάλειας (security parameters) εμφανίζονται στις αποδείξεις ασφάλειας κρυπτογραφικών σχημάτων που δεν διαθέτουν τέλεια μυστικότητα, ως ένας τρόπος μέτρησης της δυσκολίας του αντιπάλου να σπάσει την ασφάλεια τους. Στην βιβλιογραφία απαντάμε δύο κύριες παραμέτρους ασφάλειας, της στατιστικής και της υπολογιστικής.

\begin{definition}
\textbf{Παράμετρος στατιστικής ασφάλειας $σ$}, είναι ένα μέτρο της πιθανότητας που έχει ο αντίπαλος να σπάσει την ασφάλεια του συστήματος ή του αλγορίθμου και εξαρτάται άμεσα με την φύση αυτών.
\end{definition}

\begin{definition}
\textbf{Παράμετρος υπολογιστικής ασφάλειας $λ$} : Είναι η παράμετρος που σχετίζεται με την δυσκολία σπασίματος προβλημάτων από τον αντίπαλο μέσω της υπολογιστικής ισχύς του. Συνήθως αναφέρεται ως το μήκος κλειδιού που χρησιμοποιείται σε ένα κρυπτοσύστημα και όταν δίνεται ως όρισμα σε μια συνάρτηση συμβολίζεται με τον μοναδιαίο συμβολισμό $1^λ$.
\end{definition}

\begin{definition}
\textbf{Μηδαμινή συνάρτηση (Negligible function)} ονομάζουμε κάθε συνάρτηση $negl: \mathbb{N} \rightarrow \mathbb{R}$ που τείνει ως προς το $0$ ασυμπτωτικά γρηγορότερα από κάθε αντίστροφη πολυωνυμική συνάρτηση. Ή ισοδύναμα, κάθε συνάρτηση $negl$ της οποίας η αντίστροφη τείνει στο $\infty$ ασυμπτωτικά γρηγορότερα από κάθε πολυωνυμική συνάρτηση.
\end{definition}
Συνήθως στις αποδείξεις ασφάλειας η μηδαμινή συνάρτηση παίρνει ως είσοδο κάποια παράμετρο ασφάλειας, οπότε στη βιβλιογραφία πολύ συχνά συναντάμε συμβολισμούς όπως ο $negl(1^λ)$.

Ας υποθέσουμε ότι έχουμε δύο πιθανοτικούς αλγορίθμους, που η έξοδος του ακολουθεί κατανομή $D_1$ και $D_2$ αντίστοιχα. 
\begin{definition}
Οι κατανομές $D_1$ και $D_2$ έχουν Στατιστική δυσδιακριτότητα (Indistinguishable distributions), αν για οποιονδήποτε αλγόριθμο $A$ ισχύει:
$$
\Delta\left(D_{1}(n), D_{2}(n)\right)=\frac{1}{2} \sum_{x}\left|\operatorname{Pr}\left[x=D_{1}(n)\right]-\operatorname{Pr}\left[x=D_{2}(n)\right]\right|
$$
όπου $Δ$ είναι η στατιστική απόσταση των δύο κατανομών, όπου $n$ μια παράμετρος ασφάλειας.
\end{definition}

\begin{definition}
Οι κατανομές $D_1$ και $D_2$ έχουν Υπολογιστική δυσδιακριτότητα (Computational indistinguishability), αν για οποιονδήποτε μη ομοιόμορφο πιθανοτικό πολυωνυμικό αλγόριθμο (non-uniform PPT) $A$ ισχύει:
$$
\operatorname{Pr}\left[A\left(D_{1}(n)\right)=1\right]-\operatorname{Pr}\left[A\left(D_{2}(n)\right)=1\right] \leq negl(n)
$$
όπου $n$ μια παράμετρος ασφάλειας.
\end{definition}

\subsection{Βασικές υποθέσεις}

\begin{definition}
\textbf{Συνάρτηση Μονής Διαδρομής (One Way Function ή OWF)} : Μια συνάρτηση  $f : \bin^{*} → \bin^{*}$ η οποία είναι εύκολο να υπολογιστεί από έναν πολυωνυμικό στο χρόνο αλγόριθμο αλλά η αντίστροφη της, η $f^{-1}$ δεν μπορεί να υπολογιστεί επιτυχώς με μη μηδαμινή πιθανότητα από οποιονδήποτε πολυωνυμικό αλγόριθμο.
\end{definition}

\begin{definition}
\textbf{Υπόθεση Διακριτού Λογαρίθμου (Discrete Logarithm Assumption)} : Έστω μια κυκλική πολλαπλασιαστική ομάδα $\langle G, \cdot \rangle$  με $G = \langle g \rangle$ όπου $g$ ένας γεννήτορας της $G$ (π.χ. η $\langle \mathbb{Z}_p, \cdot \rangle$ όπου  $p$ πρώτος αριθμός). Η υπόθεση αυτή εκφράζει ότι, οποιοσδήποτε πολυωνυμικός στον αριθμό ψηφίων της τάξης της ομάδας αλγόριθμος, για ένα τυχαίο στοιχείο $a \in G$ έχει μηδαμινή πιθανότητα επίλυσης του Προβλήματος Διακριτού Λογαρίθμου, δηλαδή την εύρεση $x \in G$ τέτοιου ώστε $log_g(a) = x$.
\end{definition}

\begin{definition}
\textbf{Υπόθεση Αποφαστιστικού Diffie-Hellman (Decisional Diffie-Hellman ή DDH Assumption)} :
Έστω μια πολλαπλασιαστική ομάδα $G$, με τάξη $|G|=p$. Η DDH υπόθεση εκφράζει ότι οι πιθανοτικές κατανομές $(g^{a},g^{b},g^{ab})$ και $(g^{a},g^{b},g^{c})$, όπου $a, b, c \in G$ είναι τυχαία και ομοιόμορφα επιλεγμένα, είναι υπολογιστικά δυσδιάκριτες. Ισοδύναμα, το πλεονέκτημα ενός αντιπάλου με υπολογιστικά περιορισμένους πόρους που προσπαθεί να τις διακρίνει είναι το πολύ μηδαμινό.
\end{definition}

\subsection{Συμμετρική Κρυπτογραφία}

Η συμμετρική κρυπτογραφία παίρνει το όνομά της από την συμμετρική χρήση κλειδιών, δηλαδή το ίδιο κλειδί που θα χρησιμοποιηθεί για την κρυπτογράφηση θα πρέπει να χρησιμοποιηθεί και για την αποκρυπτογράφηση. Στην κατηγορία των συμμετρικών κρυπτοσυστημάτων ανήκουν όλα τα σχήματα της "Κλασσικής Κρυπτογραφίας".

\begin{definition}
\textbf{Συμμετρικό Κρυπτογραφικό Σχήμα (Symmetric Cryptographic Scheme)} : Αποτελείτε από ένα σύνολο αλγορίθμων $(\textbf{KGen}, \textbf{Enc}, \textbf{Dec})$ με τις εξής λειτουργίες :
\begin{itemize}
    \item $k \leftarrow \textbf{KGen}(1^\secpar)$ : Δημιουργεί ένα κρυπτογραφικό κλειδί $k$ το οποίο χρησιμοποιείται τόσο για κρυπτογράφηση όσο και για αποκρυπτογράφηση.
    \item $c \leftarrow \textbf{Enc}_{pk}(m)$ : Κρυπτογραφεί ένα μήνυμα $m$ χρησιμοποιώντας το κλειδί $k$.
    \item $m \leftarrow \textbf{Dec}_{sk}(c)$ : Αποκρυπτογραφεί ένα κρυπτοκείμενο $c$ χρησιμοποιώντας το κλειδί $k$.
\end{itemize}
Οι παραπάνω αλγόριθμοι πρέπει να ακολουθούν την εξής ιδιότητα ορθότητας :
$$
    Pr(\textbf{Dec}_{k}(\textbf{Enc}_{k}(m)) = 1
$$
\end{definition}

Ένα από τα βασικότερα μειονεκτήματα των συμμετρικών κρυπτογραφικών σχημάτων είναι ότι στην περίπτωση που χρησιμοποιούνται για ασφαλή επικοινωνία για κάθε συμμετέχοντα που θέλει να συμμετάσχει στην επικοινωνία θα πρέπει ο κάθε συμμετέχοντας να διαθέτει για την επικοινωνία τους ένα μοναδικό κλειδί. Δηλαδή στην περίπτωση των $n$ συμμετεχόντων θα χρειαστούμε $n^2$ κλειδιά, το πρόβλημα αυτό είναι γνωστό και ως "Πρόβλημα των Τετραγώνων" στην βιβλιογραφία. Έτσι τα σχήματα αυτά αυτοτελή καθίστανται ανίκανα για χρήση σε μεγάλα δίκτυα επικοινωνίας όπως το Διαδίκτυο. Ένα ακόμη μειονέκτημα είναι ότι οι συμμετέχοντες θα πρέπει να έχουν προαποφασίσει ένα συμμετρικό το συμμετρικό τους κλειδί. Τα προβλήματα αυτά έλυσε η εφεύρεση της ασύμμετρης κρυπτογραφίας. Τέλος σημαντικό είναι να αναφέρουμε ότι αυτά τα κρυπτογραφικά σχήματα λόγω του μεγάλου τους πλεονεκτήματος να έχουν πολύ γρήγορες υλοποιήσεις χρησιμοποιούνται κατά κόρον για μεταφορά μεγάλου όγκου δεδομένων.

\subsection{Ασύμμετρη Κρυπτογραφία}

Η Ασύμμετρη Κρυπτογραφία αναφέρθηκε πρώτη φοράς ως όρος στη βιβλιογραφία από τους Diffie-Hellman στην εργασία \improvement{Improve and add reference} και προτάθηκε ως ένα κρυπτογραφικό εργαλείο που θα βοηθούσε στην αντιστάθμιση των μειονεκτημάτων των συμμετρικών σχημάτων. Ωστόσο το πρώτο ευρέως γνωστό πρωτόκολλο ασύμμετρης κρυπτογραφίας παρουσιάστηκε στην βιβλιογραφία από τους Ronald, Shamir, Ademire και πήρε το όνομα του από τα αρχικά των δημιουργών του, ήταν το πασίγνωστο πλέον RSA. 

\begin{definition}
\textbf{Ασυμμετρικό Κρυπτογραφικό Σχήμα (Assymetric Cryptographic Scheme)} : Αποτελείτε από ένα σύνολο αλγορίθμων $(\textbf{KGen}, \textbf{Enc}, \textbf{Dec})$ με τις εξής λειτουργίες :
\begin{itemize}
    \item $(pk, sk) \leftarrow \textbf{KGen}(1^\secpar)$ : Δημιουργεί ένα ζεύγος ασύμμετρων κρυπτογραφικά κλειδιών. Το Δημόσιο Κλειδί (Public Key) $pk$ που χρησιμοποιείται στην κρυπτογράφηση και το αντίστοιχο Ιδιωτικό Κλειδί (Private Key) $sk$ που χρησιμοποιείται στην αποκρυπτογράφηση.
    \item $c \leftarrow \textbf{Enc}_{pk}(m)$ : Κρυπτογραφεί ένα μήνυμα $m$ χρησιμοποιώντας το δημόσιο κλειδί $pk$.
    \item $m \leftarrow \textbf{Dec}_{sk}(c)$ : Αποκρυπτογραφεί ένα κρυπτοκείμενο $c$ χρησιμοποιώντας το ιδιωτικό κλειδί $sk$.
\end{itemize}
Όπως και στην περίπτωση της Συμμετρικής Κρυπτογραφίας οι παραπάνω αλγόριθμοι πρέπει να ακολουθούν την εξής ιδιότητα ορθότητας :
$$
    Pr(\textbf{Dec}_{sk}(\textbf{Enc}_{pk}(m)) = 1
$$
\end{definition}

Η Ασύμμετρική Κρυπτογραφία λύνει το πρόβλημα της διανομής κρυπτογραφικών κλειδιών που υπάρχει στην περίπτωση της Συμμετρικής, καθότι ο κάθε συμμετέχων χρειάζεται να διαθέτει μόνο ένα ζέυγος κλειδιών. Ωστόσο, τα συμμετρικά κρυπτογραφικά σχήματα είναι πιο κοστοβόρα από άποψη υπολογισμών και πόρων στις υλοποιήσεις τους. Έτσι, σήμερα αρκετά δημοφιλή κρυπτογραφικά πρωτόκολλα, όπως για παράδειγμα το πρωτόκολλο TLS, που χρησιμοποιείται για την ασφαλή επικοινωνία στο Επίπεδο Μεταφοράς ενός δικτύου, στην αρχή της συνεδρίας χρησιμοποιούνται ασυμμετρικά σχήματα ψηφιακών υπογραφών και ανταλλαγής κλειδιών, όπως τα ECDSA και ECDHE αντίστοιχα, με σκοπό την δημιουργία από κοινού κλειδιών που χρησιμοποιούνται στην συνέχεια από συμμετρικούς αλγορίθμους για την κρυπτογράφηση των δεδομένων της επακόλουθης συνεδρίας. Ένα από τα κύρια χαρακτηριστικά των ασυμμετρικών σχημάτων είναι ότι πιο μαθηματικά δομημένα. Αυτό έχει ως αποτέλεσμα συνήθως να βασίζονται έμεσσα σε πιο πολύπλοκους μαθηματικούς υπολογισμούς, όπως η εύρεση μεγάλων πρώτων αριθμών σε σχέση με τους σχετικά απλούς υπολογισμούς που εκτελή ένα συμμετρικό σχήμα. Μια ακόμα συνέπεια αυτού του χαρακτηριστικού είναι ότι η ασφάλεια τους κατά κόρον βασίζεται σε υποθέσεις πολυπλοκότητας, όπως αυτές που αναφέραμε παραπάνω, από τις οποίες πολλές από αυτές έχει αποδειχθεί ότι δεν ισχύουν σε μη κλασσικά μοντέλα υπολογισμού, όπως το κβαντικό.

\subsection{Αποδείξεις Μηδενικής Γνώσης}



\subsection{Λοιπά Κρυπτογραφικά Εργαλεία}

\begin{definition}
\textbf{Πρωτόκολλο Ανυποψίαστης Μεταφοράς 1/2 (Oblivious Transfer 1/2)} : 
\begin{itemize}
\item \textbf{Εισόδοι} : 

\begin{itemize}
\item Κοινή είσοδος : Οικογένεια Trapdoor Permutations (TDP) : $(I, S, F, F^{-1})$, Hardcore Predicate : $B$
\item $P_{1}$ : $b_{0}, b_{1} \in\{0,1\}$
\item $P_{2}$ : $\sigma \in\{0,1\}$.
\end{itemize}

\item \textbf{Πρωτόκολλο} :
\begin{enumerate}
\item $P_{1}$ runs $(\alpha, \tau) \leftarrow I\left(1^{n}\right)$. $P_{1}$ sends $\alpha$ to $P_{2}$.
\item $P_{2}$ runs $S(\alpha)$ twice: denote the first value obtained by $x_{\sigma}$ and the second by $y_{1-\sigma}$. Then, $P_{2}$ computes $y_{\sigma}=F\left(\alpha, x_{\sigma}\right)=f_{\alpha}\left(x_{\sigma}\right)$, and sends $y_{0}, y_{1}$ to $P_{1}$.
\ite  $P_{1}$ uses the trapdoor $\tau$ and computes $x_{0}=F^{-1}\left(\alpha, y_{0}\right)=f_{\alpha}^{-1}\left(y_{0}\right)$ and $x_{1}=$ $F^{-1}\left(\alpha, y_{1}\right)=f_{a}^{-1}\left(y_{1}\right)$. Then, it computes $\beta_{0}=B\left(\alpha, x_{0}\right) \oplus b_{0}$ and $\beta_{1}=B\left(\alpha, x_{1}\right) \oplus b_{1}$, where $B$ is a hard-core predicate of $f$. Finally, $P_{1}$ sends $\left(\beta_{0}, \beta_{1}\right)$ to $P_{2}$.
\item $P_{2}$ computes $b_{\sigma}=B\left(\alpha, x_{\sigma}\right) \oplus \beta_{\sigma}$ and outputs the result.
\end{enumerate}

\item \textbf{Έξοδος}
$P_{2} : b_σ$
\end{itemize}
\end{definition}

Ας υποθέσουμε ότι ο ιδιοκτήτης μιας επιχείρησης έχει στην κατοχή του κάποια πολύ σημαντικά έγγραφα τα οποία βρίσκονται σε κάποια θυρίδα που είναι ασφαλισμένη με κάποιο κωδικό. Σε αυτά τα έγγραφα θέλει να έχει πρόσβαση το Διοικητικό Συμβούλιο της εταιρίας ακόμα και στην περίπτωση που αυτός δεν είναι παρόν, π.χ. στην περίπτωση που αυτός ξαφνικά φύγει από την ζωή, υπό την προϋπόθεση όμως ότι πάνω από το $50\%+1$ των συμβούλων συμφωνεί στο άνοιγμα της θυρίδας. 

Το Πρωτόκολλο Διαμοίρασης Μυστικών Shamir ουσιαστικά λύνει αυτό το πρόβλημα. Βασίζεται στο γεγονός ότι οποιοδήποτε σύνολο $n+1$ σημείων σε ένα δυδιάστατο πεδίο $\mathbf{F}^2$ καθορίζει μοναδικά ένα πολυώνυμο $n$ βαθμού. Δηλαδή αν $f:\mathbbb{F} \rightarrow \mathbb{F}$ όπου $f(x)=\pn{α}{x}$ τότε ένα σύνολο σημείων $(x_i, f(x_i))$ όπου $x \in [1, n+1]$ είναι επαρκές για να ανακτήσουμε την συνάρτηση $f$. Αυτό μπορεί πολύ εύκολα να συμβεί λύνοντας το σύστημα εξισώσεων μεγέθους $(n+1) \ctimes (n+1)$ που προκύπτει αν αντικαταστήσουμε με κάθε σημείο την γενική εξίσωση πολυωνύμου $f(x_i)=\pn{a}{x_i}$. Ένας πιο γρήγορος τρόπος να επιτευχθεί αυτό είναι μέσω των πολλαπλασιαστών Lagrange που αναλήθηκαν στην Ενότητα \improvement{reference}.

\begin{definition}
\textbf{Πρωτόκολλο Διαμοίρασης Μυστικών Shamir (Shamir's Secret Sharing ή SSS)} : Το πρωτόκολλο αυτό χρησιμοποιείται για την δοιαμοίραση ενός μυστικού σε $t$ συμμετέχοντες με τρόπο έτσι ώστε να χρειάζεται τουλάχιστον $n$, όπου $t \gte n$, να συνεργαστούν ώστε να γίνει η αποκάλυψη του μυστικού.

\begin{itemize}

\item \textbf{Εισόδοι} :
\begin{itemize}
    \item Κοινή είσοδος : Πεδίο \mathbb{F}
    \item Διαμοιραστής : Μυστικό $s \in \mathbb{F}$
\end{itemize}

\item \textbf{Πρωτόκολλο}
\begin{enumerate}
    \item Initially, in order to encrypt the secret code, we build a polynomial of degree (K – 1).
    \item Therefore, let the polynomial be y = a + bx. Here, the constant part ‘a’ is our secret code.
    \item Let b be any random number, say b = 15.
    \item Therefore, for this polynomial y = 65 + 15x, we generate N = 4 points from it.
    \item Let those 4 points be (1, 80), (2, 95), (3, 110), (4, 125). Clearly, we can generate the initial polynomial from any two of these 4 points and in the resulting polynomial, the constant term a is the required secret code.
\end{enumerate}

\item \textbf{Εξόδοι}

\end{itemize}
\end{definition}

Αξίζει να σημειωθεί ότι στην βιβλιογραφία έχουν προταθεί αρκετά πρωτόκολλα για Διαμοίραση Μυστικών. Όπως του Blakley, του Mignotte και των Asmut-Bloom. Το πρώτο βασίζεται στο γεγονός ότι αν πάρουμε $n$ $n$-διάστατα υπερεπιπέδα με την προϋπόθεση ότι ανά δύο δεν είναι παράλληλα τότε όλα έχουν ένα κοινό μονοδιάστατο κοινό σημείο τομής. Έτσι αν θέλουμε να πάρουμε ένα σχήμα Διαμοίρασης Μυστικών για $t-n$ συμμετέχοντες τότε αρκεί να δώσουμε 